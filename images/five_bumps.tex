%!TEX TS-program = xelatex
%!TEX encoding = UTF-8 Unicode

\documentclass[11pt,tikz,border=1]{standalone}
\usetikzlibrary{calc,positioning}
\usepackage{pgfplots}

\usepackage[mdseries=Light,bfseries=Medium,path=../fonts]{cjkfonts}

\begin{document}

  \begin{tikzpicture}[
    neuron/.style={circle,draw,inner sep=0pt,minimum size=10mm}
    ]

    \begin{scope}
      
      \node (l0) [neuron] {$x$};
      \node (m0g0) [neuron,right=1.5 of l0,yshift=-5.5mm] {$0.6$};
      \node (m1g0) [neuron,right=1.5 of l0,yshift=5.5mm] {$0.4$};
      \node (r0) [neuron,right=1.5 of m0g0,yshift=5.5mm] {};

      \node (m0g1) [neuron,above=0.4 of m1g0] {$0.4$};
      \node (m1g1) [neuron,above=1mm] at (m0g1.north) {$0.2$};

      \node (m1g2) [neuron,below=0.4 of m0g0] {$0.6$};
      \node (m0g2) [neuron,below=1mm] at (m1g2.south) {$0.8$};

      \node (m0g3) [neuron,above=0.4 of m1g1] {$0.2$};
      \node (m1g3) [neuron,above=1mm] at (m0g3.north) {$0.0$};

      \node (m1g4) [neuron,below=0.4 of m0g2] {$0.8$};
      \node (m0g4) [neuron,below=1mm] at (m1g4.south) {$1.0$};

      \foreach \x in {0,1}
        \foreach \y in {0,...,4} {
          \draw[->] (l0) to (m\x g\y);
          \draw[->] (m\x g\y) to (r0);
      }

      \draw[->] (r0) -- ++(1,0);

      \draw[->] (l0) to (m0g0);
      \draw[->] (l0) to (m1g0);
      \draw[->] (m0g0) to (r0);
      \draw[->] (m1g0) to (r0);

%      \node(b) [blue,above] at (m1g0.north) {$s = 0.49$};
    \end{scope}

    \begin{scope}[xshift=7cm]

      \begin{axis}[
          width=5.5cm,
          height=5.5cm,
          xlabel={\normalsize $x$},
          axis lines=left,
          tick label style={font=\tiny},
          label style={font=\tiny},
          title style={font=\scriptsize},
          % xtick={0,1},
          % ytick={0,1},
          % minor tick num=1,
          title={隐藏层的带权输出}
        ]
        \addplot[
          orange,
          thick,
          domain=0:1,
          samples=101
        ] {
          1/(1 + exp(-(100 * x + (-0.49 * 100))))        
        };
      \end{axis}
      
    \end{scope}
    
  \end{tikzpicture}

\end{document}
